\chapter*{Практическая часть}
Составить программу, т.е. модель предметной области – базу знаний, объединив в ней
информацию – знания:
\begin{itemize}
	\item «Телефонный справочник»: Фамилия, №тел, Адрес – структура (Город,
	Улица, №дома, №кв);
	\item «Автомобили»: Фамилия, Марка, Цвет, Стоимость;
	\item «Вкладчики банков»: Фамилия, Банк, счет, сумма.
\end{itemize}

Владелец может иметь несколько телефонов, автомобилей, вкладов (Факты).
Используя правила, обеспечить возможность поиска:
\begin{enumerate}
	\item по номеру телефона найти: Фамилию, Марку автомобиля, Стоимость автомобиля
	(может быть несколько);
	\item используя сформированное в пункте 1 правило, по номеру телефона найти:
	только марку автомобиля (автомобилей может быть несколько);
	\item используя простой, не составной вопрос: по Фамилии (уникальна в городе, но в
	разных городах есть однофамильцы) и Городу проживания найти: улицу проживания, банки, в которых есть вклады и номеру телефона.
\end{enumerate}

Используя конъюнктивное правило и простой вопрос, обеспечить возможность поиска по марке и цвету автомобиля фамилию владельца, его город, телефон и банки, где у владельца есть вклады.

Описать словесно порядок поиска ответа на вопрос, указав, как выбираются знания, и, при этом, для
каждого этапа унификации, выписать подстановку – наибольший общий унификатор, и соответствующие примеры термов.

На листинге 1 представлен код программы:

\FloatBarrier
\begin{lstinputlisting}[style={lsp}]{src/main.pro}
\end{lstinputlisting}
\FloatBarrier

В приложении 1 приведены таблицы для описания порядка ответа на вопрос, как выбираются знания.

\chapter*{Теоретические вопросы}
\section*{1. Что такое терм?}

Терм - основной элемент языка \texttt{Prolog}. Терм – это:
\begin{enumerate}
	\item Константа: 
	\begin{itemize}
		\item Число (целое, вещественное),
		\item Символьный атом (комбинация символов латинского алфавита, цифр и символа подчеркивания, начинающаяся со строчной буквы),
		\item Строка: последовательность символов, заключенных в кавычки.
	\end{itemize}
	\item Переменная:
	\begin{itemize}
		\item Именованная – обозначается комбинацией символов латинского алфавита, цифр и символа подчеркивания, начинающейся с прописной буквы или символа подчеркивания,
		\item Анонимная  - обозначается символом подчеркивания
	\end{itemize}
	\item Составной терм:
	Это средство организации группы отдельных элементов знаний в единый  объект,  синтаксически представляется: f(t1, t2, …,tm), где f -  функтор (отношение между объектами), t1, t2, …,tm – термы, в том  числе  и составные.
\end{enumerate}

\section*{2. Что такое предикат в матлогике (математике)?}

Предикат (\textit{n}-местный, или \textit{n}-арный) --- это функция с множеством значений \texttt{\{0, 1\}} (или \texttt{\{ложь, истина\}}), определённая на множестве $M^n = (M_1, M_2, \ldots, M_n)$. Таким образом, каждый набор элементов множества характеризуется либо как <<истинный>>, либо как <<ложный>>.

\section*{3. Что описывает предикат в \texttt{Prolog}?}

Предикат в \texttt{Prolog} описывает отношение между аргументами процедуры. Процедурой в \texttt{Prolog} является совокупность всех правил, описывающих определенное отношение.

\section*{4. Назовите виды предложений в программе и приведите примеры таких предложений из Вашей программы. Какие предложения являются основными, а какие – не основными?  Каковы: синтаксис и семантика (формальный смысл) этих предложений (основных и неосновных)?}

В \texttt{Prolog} есть два типа предложений: правила и факты. Правило имеет вид: \texttt{A :- B1, \ldots, Bn.} 
\texttt{A} называется заголовком правила, а \texttt{B1, \ldots, Bn} – телом правила. Заголовок содержит некоторое знание, а тело --- условие истинности этого знания. Факт является частным случаем правила --- в нем отсутствует тело.

Пример факта из программы: \texttt{phoneRecord("Prianishnikov", 890801, address("Moscow", "Izmailovky", 73, 628).}


Пример правила из программы: \texttt{findCarModelByPhone(PhoneNumber, CarModel) :- findCarByPhone(PhoneNumber, \_, CarModel, \_, \_).}

Основными называются предложения, не содержащие переменных. Предложения, содержащие переменные называются неосновными. 

Синтаксис предложения: \texttt{заголовок (составной терм) :- тело (один или последовательность термов).} 

Предложения используются для формирования базы знаний о некоторой предметной области. Основное предложение описывает отношение конкретных объектов предметной области. Неосновное предложение описывает множество отношений, потому что переменная, входящая в предложение базы знаний, рассматривается только с квантором всеобщности.

\section*{5. Каковы назначение, виды и особенности использования переменных в программе на \texttt{Prolog}? Какое предложение БЗ сформулировано в более общей – абстрактной форме: содержащее или не содержащее переменных?}

Переменные предназначены для обозначения некоторого неизвестного объекта предметной области. Переменные бывают именованными и анонимными. Именованные переменные уникальны в рамках предложения, а анонимная переменная – любая уникальна. В разных предложениях может использоваться одно имя переменной для обозначения разных объектов.

В ходе выполнения программы выполняется связывание переменных с различными объектами, этот процесс называется конкретизацией. Это относится только к именованным переменным. Анонимные переменные не могут быть связаны со значением.

В более общей форме сформулировано предложение, содержащее переменные, так как заранее неизвестно, каким объектом будет конкретизирована переменная.

\section*{6. Что такое подстановка?}

Пусть дан терм: $А(X_1, X_2, \ldots , X_n)$.
Подстановка --- множество пар, вида: \\ $\{X _ i = t _ i\}$, где $X_i$ --- переменная, а $t_i$ --- терм.

\section*{7. Что такое пример терма? Как и когда строится? Как Вы думаете, система строит и хранит примеры?}

Пусть $\Theta =  \{X_1 = t_1, X_2= t_2, \dots , X_n = t_n \}$   –   подстановка, $A$ - терм. Результат применения подстановки к терму обозначается $A\Theta$.

Примером терма $A$ называется терм $B$, если существует подстановка $\Theta$ такая, что $B = A\Theta$.

Примеры термов строятся в ходе логического вывода. Для построения примера терма его переменные конкретизируются.