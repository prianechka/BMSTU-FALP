\chapter*{Практическая часть}

Создать базу знаний «Собственники», дополнив (и минимально изменив) базу знаний, хранящую знания:
\begin{enumerate}
	\item \textbf{«Телефонный справочник»:} фамилия, номер телефона, адрес – структура (город, улица, № дома, № кв).
	\item \textbf{«Автомобили»:} фамилия владельца, марка, цвет, стоимость.
	\item \textbf{«Вкладчики банков»:} Фамилия владельца, банк, счет, сумма.
\end{enumerate}

Описать и использовать вариантный домен: Собственность. 
Владелец может иметь, но только один объект каждого вида собственности (это касается и автомобиля), 
или не иметь некоторых видов собственности.

Преобразовать знания об автомобиле к форме знаний о собственности.
Виды собственности (кроме автомобиля):
\begin{enumerate}
	\item \textbf{Строение:} стоимость.
	\item \textbf{Участок:} стоимость.
	\item \textbf{Водный транспорт:} стоимость.
\end{enumerate}

Используя конъюнктивное правило и разные формы задания одного вопроса (пояснять для какого №задания – какой вопрос), обеспечить возможность поиска:

\begin{enumerate}
	\item Названий всех объектов собственности заданного субъекта.
	\item Названий и стоимости всех объектов собственности заданного субъекта.
	\item Разработать правило, позволяющее найти суммарную стоимость всех объектов собственности заданного субъекта.
\end{enumerate}

Для 2-го пункт и одной фамилии составить таблицу, отражающую конкретный порядок работы системы, с объяснениями порядка работы и особенностей использования
доменов (указать конкретные Т1 и Т2 и полную подстановку на каждом шаге).

На листинге 1 представлен код программы:

\FloatBarrier
\begin{lstinputlisting}[style={lsp}]{src/main.pro}
\end{lstinputlisting}
\FloatBarrier

В приложении 1 приведены таблицы для описания порядка ответа на вопрос, как выбираются знания.

\chapter*{Теоретические вопросы}
\section*{1. В каком фрагменте программы сформулировано знание? Это знание о чем на формальном уровне?}
Знания сформулированы в базе знаний. Знания на формальном уровне о предметной области.

\section*{2. Что содержит тело правила?}

Правило — это предложение, истинность которого зависит от истинности одного или нескольких предложений. 
Обычно правило содержит несколько хвостовых целей, которые должны быть истинными для того, чтобы правило было истинным.

\section*{3.Что дает использование переменных при формулировании знаний? В чем отличие формулировки знания с помощью термов с одинаковой арностью при использовании одной переменной и при использовании нескольких переменных?}

Переменная в Прологе, в отличие от алгоритмических языков программирования, обозначает объект, а не некоторую область памяти. Пролог не поддерживает механизм деструктивного присваивания, позволяющий изменять значение инициализированной переменной, как императивные языки. 

\section*{4. С каким квантором переменные входят в правило, в каких пределах переменная уникальна?}
Имя переменной в Прологе может состоять из букв латинского алфавита, цифр, знаков подчеркивания и должно начинаться с прописной буквы или знака подчеркивания. 
При этом переменные в теле правила неявно связаны квантором всеобщности. 

Областью действия переменной в Прологе является одно предложение. 
В разных предложениях может использоваться одно имя переменной для обозначения разных объектов.

\section*{5. Какова семантика (смысл) предложений раздела DOMAINS? Когда, где и с какой целью используется это описание?}
Раздел описания доменов является аналогом раздела описания типов в обычных императивных языках программирования и начинается с ключевого слова DOMAINS.

Если природа или структура объектов, обозначенных аргументами, между которыми устанавливается отношение в заголовке правил процедуры, не может быть определена с помощью стандартных доменов, то требуется определить имена и семантику – смысл (структуру) таких доменов в разделе DOMAINS.

\section*{6. Какова семантика (смысл) предложений раздела PREDICATES? Когда, и где используется это описание? С какой целью?}
В разделе, озаглавленном зарезервированным словом PREDICATES, содержатся описания определяемых пользователем предикатов. В традиционных языках программирования подобными разделами являются разделы описания заголовков процедур и функций.

Если природа или структура объектов, обозначенных аргументами, между которыми устанавливается отношение в заголовке правил процедуры, важна во время работы системы, то она должна быть указана в разделе PREDICATES с помощью соответствующих доменов. 

\section*{7. Унификация каких термов запускается на самом первом шаге работы системы? Каковы назначение и результат использования алгоритма унификации?}
На первом шаге работы запускается унификация вопроса и первого знания в базе знаний.

Алгоритм унификации – основной шаг с помощью которого система отвечает на вопросы унификации. Для
нахождения всех решений реализуется механизм возврата.

Унификация – необходима для того, чтобы определить дальнейший путь поиска решений. На первом шаге:
терм вопроса унифицируется с заголовком. Унификация заканчивается конкретизацией части переменных.

Алгоритм унификации принимает на вход два терма, и возвращает флаг успешности подстановки, и, если успешно, то подстановку.

\section*{8. В каком случае запускается механизм отката?}
Механизм отката, который осуществляет откат программы к той точке, в которой выбирался унифицирующийся с последней подцелью дизъюнкт.
Для этого точка, где выбирался один из возможных унифицируемых с подцелью дизъюнктов, запоминается в специальном стеке, для последующего возврата к ней и выбора альтернативы в случае неудачи. При откате все переменные, которые были означены в результате унификации после этой точки, опять становятся свободными.