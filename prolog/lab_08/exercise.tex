\chapter*{Практическая часть}
Используя хвостовую рекурсию, разработать, комментируя аргументы,
эффективную программу, позволяющую:
\begin{enumerate}
	\item Сформировать список из элементов числового списка, больших заданного значения;
	\item Сформировать список из элементов, стоящих на нечетных позициях исходного списка (нумерация от 0);
	\item Удалить заданный элемент из списка (один или все вхождения);
	\item Преобразовать список в множество (можно использовать ранее разработанные
	процедуры).
\end{enumerate}
На листинге 1 представлен код программы:

\FloatBarrier
\begin{lstinputlisting}[style={lsp}]{src/main.pl}
\end{lstinputlisting}
\FloatBarrier

\chapter*{Теоретические вопросы}

\section*{1. Что такое рекурсия? Как организуется хвостовая рекурсия в Prolog? Как организовать выход из рекурсии в Prolog?}
Рекурсия – это один из способов организации повторных вычислений. Рекурсия – это способ заставить систему использовать многократно одну и ту же процедуру.

Для организации хвостовой рекурсии рекурсивный вызов должен быть последней подцелью и нужно избавиться от точек возврата с помощью отсечения, чтобы исключить возможные альтернативы.

\section*{2. Какое первое состояние резольвенты?}
Стек, который содержит конъюнкцию целей, истинность которых система должна доказать, называется
резольвентой. Первое состояние резольвенты - вопрос.

\section*{3. В каком случае система запускает алгоритм унификации? (Как эту необходимость на формальном уровне распознает система?)}
Унификация – необходима для того, чтобы определить дальнейший путь поиска решений. Унификация заканчивается конкретизацией части переменных.

\section*{4. Каковы назначение и результат использования алгоритма унификации?}
Алгоритм унификации – основной шаг с помощью которого система отвечает на вопросы унификации. 
На вход алгоритм принимает два терма, возвращает флаг успешности унификации, и если успешно, то подстановку.

\section*{5. В каких пределах программы уникальны переменные?}
Областью действия переменной в Прологе является одно предложение.
В разных предложениях может использоваться одно имя переменной для обозначения разных объектов.

\section*{6. В каких случаях запускается механизм отката?}
Механизм отката, который осуществляет откат программы к той точке, в которой выбирался унифицирующийся с последней подцелью дизъюнкт.
Для этого точка, где выбирался один из возможных унифицируемых с подцелью дизъюнктов, запоминается в специальном стеке, для последующего возврата к ней и выбора альтернативы в случае неудачи. 
При откате все переменные, которые были означены в результате унификации после этой точки, опять становятся свободными.