\chapter*{Практическая часть}

Для каждого случая пункта 2 обосновать необходимость всех условий тела.
Для одного из вариантов ВОПРОСА и каждого варианта задания 2 составить
таблицу, отражающую конкретный порядок работы системы:
Т.к. резольвента хранится в виде стека, то состояние резольвенты требуется отображать
в столбик: вершина – сверху! Новый шаг надо начинать с нового состояния резольвенты!

\begin{enumerate}
	\item Максимум из двух чисел без использования отсечения и с использованием отсечения.
	\item Максимум из трёх чисел без использования отсечения и с использованием отсечения.
\end{enumerate}

Для каждого случая пункта 2 обосновать необходимость всех условий тела.
Для одного из вариантов ВОПРОСА и каждого варианта задания 2 составить таблицу, отражающую конкретный порядок работы системы.

Так как резольвента хранится в виде стека, то состояние резольвенты требуется отображать в столбик: вершина – сверху! Новый шаг надо начинать с нового состояния резольвенты!

На листинге 1 представлен код программы:

\FloatBarrier
\begin{lstinputlisting}[style={lsp}]{src/main.pro}
\end{lstinputlisting}
\FloatBarrier

В приложении 1 приведены таблицы для описания порядка ответа на вопрос, как выбираются знания.

\chapter*{Теоретические вопросы}

\section*{1. Какое первое состояние резольвенты?}
Стек, который содержит конъюнкцию целей, истинность которых система должна доказать, называется
резольвентой. Первое состояние резольвенты - вопрос.

\section*{2. В каком случае система запускает алгоритм унификации? (Как эту необходимость на формальном уровне распознает система?)}
Унификация – необходима для того, чтобы определить дальнейший путь поиска решений. Унификация заканчивается конкретизацией части переменных.

\section*{3. Каковы назначение и результат использования алгоритма унификации?}
Алгоритм унификации – основной шаг с помощью которого система отвечает на вопросы унификации. 
На вход алгоритм принимает два терма, возвращает флаг успешности унификации, и если успешно, то подстановку.

\section*{4. В каких пределах программы уникальны переменные?}
Областью действия переменной в Прологе является одно предложение.
В разных предложениях может использоваться одно имя переменной для обозначения разных объектов.


\section*{5. Как применяется подстановка, полученная с помощью алгоритма унификации?}
Пусть дан терм: $А(X_1, X_2, \ldots , X_n)$.
Подстановка --- множество пар, вида: \\ $\{X _ i = t _ i\}$, где $X_i$ --- переменная, а $t_i$ --- терм.

В ходе выполнения программы выполняется связывание переменных с различными объектами, этот процесс назыв

\section*{6. Как меняется резольвента?}
Резольвента меняется в 2 этапа:
\begin{enumerate}
	\item Редукция – замена подцели телом того правила, с заголовком которого успешно унифицируется данная подцель
	\item Применение ко всей резольвенте подстановки.
\end{enumerate}

Резольвента уменьшается, если удаётся унифицировать подцель с фактом. Система отвечает «Да», только
когда резольвента становится пустой.

\section*{7. В каких случаях запускается механизм отката?}
Механизм отката, который осуществляет откат программы к той точке, в которой выбирался унифицирующийся с последней подцелью дизъюнкт.
Для этого точка, где выбирался один из возможных унифицируемых с подцелью дизъюнктов, запоминается в специальном стеке, для последующего возврата к ней и выбора альтернативы в случае неудачи. 
При откате все переменные, которые были означены в результате унификации после этой точки, опять становятся свободными.