\chapter*{Практическая часть}

\section*{Задание 1}
Запустить среду Visual Prolog 5.2. Настроить утилиту TestGoal. 
Запустить тестовую программу, проанализировать реакцию системы и множество ответов. 
Разработать свою программу - «Телефонный справочник». 
Протестировать работу программы.

При установке утилиты TestGoal добиться того, чтобы в результате выполнения
тестовой программы возвращалось несколько вариантов ответов.

На листинге 1 представлен код программы:

\FloatBarrier
\begin{lstinputlisting}[style={lsp}]{src/main.pro}
\end{lstinputlisting}
\FloatBarrier

\section*{Задание 2}
Составить программу – базу знаний, с помощью которой можно определить, например,
множество студентов, обучающихся в одном ВУЗе и их телефоны. Студент может
одновременно обучаться в нескольких ВУЗах. Привести примеры возможных вариантов
вопросов и варианты ответов (не менее 3-х). Описать порядок формирования вариантов
ответа.

На листинге 2 представлен код программы:
\FloatBarrier
\begin{lstinputlisting}[style={lsp}]{src/main2.pro}
\end{lstinputlisting}
\FloatBarrier

\chapter*{Теоретические вопросы}
Основным элементом языка Prolog является терм: константа, переменная или составной терм. 
В некоторых случаях, можно сказать, что составной терм является предикатом.

Программа на Prolog не является последовательностью действий, - она представляет
собой набор фактов и правил, которые формируют базу знаний о предметной области. 
Факты представляют собой составные термы, с помощью которых фиксируется наличие
истинностных отношений между объектами предметной области — аргументами терма.

Утверждения программы — это предикаты. Предикаты могут не содержать переменных
(основные) или содержать переменные (не основные). В процессе выполнения программы —
система пытается найти, используя базу знаний , такие значения переменных, при которых на
поставленный вопрос можно дать ответ «Да».

Программа на Prolog состоит из разделов. Каждый раздел начинается со своего
заголовка. Структура программы:
\begin{itemize}
	\item директивы компилятора — зарезервированные символьные константы
	\item CONSTANTS — раздел описания констант
	\item DOMAINS — раздел описания доменов
	\item DATABASE — раздел описания предикатов внутренней базы данных
	\item PREDICATES — раздел описания предикатов
	\item CLAUSES — раздел описания предложений базы знаний
	\item GOAL — раздел описания внутренней цели (вопроса).
\end{itemize}

В программе не обязательно должны быть все разделы.
