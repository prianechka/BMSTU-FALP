\chapter*{Практическое задание}

\section*{Задание 1}
Написать функцию, которая по своему списку-аргументу lst определяет, 
является ли он палиндромом (то есть равны ли lst и (reverse lst)).

\FloatBarrier
\begin{lstinputlisting}[style={lsp}]{src/1.lsp}
\end{lstinputlisting}
\FloatBarrier

\section*{Задание 2}
Написать предикат, который возвращает t, если два его множества-
аргумента содержат одни и те же элементы, порядок которых не имеет значения.

\FloatBarrier
\begin{lstinputlisting}[style={lsp}]{src/2.lsp}
\end{lstinputlisting}
\FloatBarrier

\section*{Задание 3}
Напишите свои необходимые функции, которые обрабатывают таблицу из 4-х точечных
пар: (страна . столица), и возвращают по стране - столицу, а по столице — страну .

\FloatBarrier
\begin{lstinputlisting}[style={lsp}]{src/3.lsp}
\end{lstinputlisting}
\FloatBarrier


\section*{Задание 4}
Напишите функцию swap-first-last, которая переставляет в списке-аргументе первый и
последний элементы.

\FloatBarrier
\begin{lstinputlisting}[style={lsp}]{src/4.lsp}
\end{lstinputlisting}
\FloatBarrier

\section*{Задание 5}
Напишите функцию swap-two-ellement, которая переставляет в списке- аргументе два
указанных своими порядковыми номерами элемента в этом списке.

\FloatBarrier
\begin{lstinputlisting}[style={lsp}]{src/5.lsp}
\end{lstinputlisting}
\FloatBarrier

\section*{Задание 6}
Напишите две функции, swap-to-left и swap-to-right, которые производят одну круговую
перестановку в списке-аргументе влево и вправо, соответственно.

\FloatBarrier
\begin{lstinputlisting}[style={lsp}]{src/6.lsp}
\end{lstinputlisting}
\FloatBarrier

\clearpage

\section*{Задание 7}
Напишите функцию, которая добавляет к множеству двухэлементных списков новый
двухэлементный список, если его там нет.

\FloatBarrier
\begin{lstinputlisting}[style={lsp}]{src/7.lsp}
\end{lstinputlisting}
\FloatBarrier


\section*{Задание 8}
Напишите функцию, которая умножает на заданное число-аргумент первый числовой
элемент списка из заданного 3-х элементного списка-аргумента, когда
\begin{itemize}
	\item все элементы списка --- числа;
	\item элементы списка -- любые объекты.
\end{itemize}

\FloatBarrier
\begin{lstinputlisting}[style={lsp}]{src/8.lsp}
\end{lstinputlisting}
\FloatBarrier


\section*{Задание 9}
Напишите функцию, select-between, которая из списка-аргумента из 5 чисел выбирает
только те, которые расположены между двумя указанными границами-аргументами и
возвращает их в виде списка (упорядоченного по возрастанию списка чисел (+ 2 балла)).

\FloatBarrier
\begin{lstinputlisting}[style={lsp}]{src/9.lsp}
\end{lstinputlisting}
\FloatBarrier

\chapter*{Теоретические вопросы}

\section*{1. Структуроразрушающие и не разрушающие структуру списка функции}

\subsection*{Не разрушающие структуру списка функции}

\begin{itemize}
	\item \texttt{append} --- Объединяет списки. Создает копию для всех аргументов, кроме последнего;
	
	\item \texttt{reverse} --- Возвращает копию исходного списка, элементы которого переставлены в обратном порядке (работает только на верхнем уровне);
	
	\item \texttt{last} --- Возвращает последнюю списковую ячейку верхнего уровня;
	
	\item \texttt{nth} --- Возвращает указателя от n-ной списковой ячейки. Нумерация начинается с нуля;
	
	\item \texttt{nthcdr} --- Возвращает n-ого хвоста;
	
	\item \texttt{length} --- Возвращает длину списка (верхнего уровня);
	
	\item \texttt{remove} --- Данная функция удаляет элемент по значению (работает с копией), можно передать функцию сравнения через \texttt{:test};
	
	\item \texttt{subst} --- Заменяет все элементы списка, которые равны 2 переданному элементу-аргументу на другой 1 элемент-аргумент.	
\end{itemize}

\subsection*{Структуроразрушающие функции}

Данные функции меняют сам объект-аргумент, невозможно вернуться к исходному списку. Чаще всего такие функции начинаются с префикса \texttt{n-}.

\begin{itemize}
	\item \texttt{nconc} --- Работает аналогично \texttt{append}, только не копирует свои аргументы, а разрушает структуру;
	
	\item \texttt{nreverse} --- Работает аналогично \texttt{reverse}, но не создает копии;
	
	\item \texttt{nsubst} --- Работает аналогично функции \texttt{nsubst}, но не создает копии;
	
\end{itemize}

\section*{2. Отличие в работе функций \texttt{cons}, \texttt{list}, \texttt{append}, \texttt{nconc} и в их результате}

Функция \texttt{cons} cons создает списковую ячейку и ставит указатели на 2 аргумента, таким образом объединяя свои аргументы в точечную пару.

Примеры:
\begin{enumerate}
	\item \texttt{(cons 1 '(2 3))} --- \texttt{(1 2 3)};
	\item \texttt{(cons '(2 3) 1)} --- \texttt{((2 3) . 1)}.
\end{enumerate}

Функция \texttt{list} принимает произвольное число аргументов. Функция возвращает список, состоящий из значений аргументов.

\begin{enumerate}
	\item[] \texttt{(list 1 2 3)} --- \texttt{(1 2 3)};
	\item[] \texttt{(list 2 '(1 2))} --- \texttt{(2 (1 2))};
	\item[] \texttt{(list '(1 2) '(3 4))} --- \texttt{((1 2) (3 4))};
\end{enumerate}

Функция \texttt{append} --- форма, принимает на вход произвольное количество аргументов и для всех аргументов, кроме последнего, создает копию, ссылая при этом последний элемент каждого списка-аргумента на первый элемент следующего по порядку списка-аргумента.

\begin{enumerate}
	\item[] \texttt{(append '(1 2) '(3 4))} --- \texttt{(1 2 3 4)};
	\item[] \texttt{(append '((1 2) (3 4)) '(5 6))} --- \texttt{((1 2) (3 4) 5 6)}.
\end{enumerate}

Функция \texttt{nconc} принимает на вход произвольное число аргументов, каждый из которых должен быть списком, кроме последнего. Функция возвращает список, который образован конкатенацией всех аргументов. Результат сохраняется не в копию массива, а в первый аргумент.

\begin{enumerate}
	\item[] \texttt{(nconc '(1 2) '(3 4) '(5 6))} --- \texttt{(1 2 3 4 5 6)};
	\item[] \texttt{(nconc '((1 2) (3 4)) '(5 6))} --- \texttt{((1 2) (3 4) 5 6)}.
\end{enumerate}