\chapter{Практическое задание}

\section*{Задание 1}
Составить диаграмму вычисления следующих выражений:

\begin{lstlisting}[style={scheme}]
1. 
(equal 3 (abs -3))
	3 %\as% 3
	(abs -3)
		-3 %\as% -3
		abs %\at% -3
		3
	equal %\at% 3 и 3
	T
\end{lstlisting}

\begin{lstlisting}[style={scheme}]
2. 
(equal (+ 1 2) 3)
	(+ 1 2)
		1 %\as% 1
		2 %\as% 2
		+ %\at% 1 и 2
		3
	3 %\as% 3
	equal %\at% 3 и 3
	T
\end{lstlisting}

\clearpage
\begin{lstlisting}[style={scheme}]
3. 
(equal (* 4 7) 21)
	(* 4 7)
		4 %\as% 4
		7 %\as% 7
		* %\at% 4 и 7
		28
	21 %\as% 21
	equal %\at% 28 и 21
	Nil
\end{lstlisting}

\begin{lstlisting}[style={scheme}]
4. 
(equal (* 2 3) (+ 7 2))
	(* 2 3)
		2 %\as% 2
		3 %\as% 3
		* %\at% 2 и 3
		6
	(+ 7 2)
		7 %\as% 7
		2 %\as% 2
		+ %\at% 7 и 2
		9
	equal %\at% 6 и 9
	Nil
\end{lstlisting}

\clearpage
\begin{lstlisting}[style={scheme}]
5. 
(equal (- 7 3) (* 3 2))
	(- 7 3)
		7 %\as% 7
		3 %\as% 3
		- %\at% 7 и 3
		4
	(* 3 2)
		3 %\as% 3
		2 %\as% 2
		* %\at% 3 и 2
		6
	equal %\at% 4 и 6
	Nil
\end{lstlisting}

\begin{lstlisting}[style={scheme}]
6. 
(equal (abs (- 2 4)) 3)
	(abs (- 2 4))
		(- 2 4)
			2 %\as% 2
			4 %\as% 4
			- %\at% 2 и 4
			-2
		abs %\at% -2
		2
	3 %\as% 3
	equal %\at% 2 и 3
	Nil
\end{lstlisting}
\clearpage
\section*{Задание 2}
Написать функцию, вычисляющую гипотенузу прямоугольного
треугольника по заданным катетам и составить диаграмму её вычисления.

\FloatBarrier
\begin{lstinputlisting}[style={lsp}]{src/2.lsp}
\end{lstinputlisting}
\FloatBarrier

\begin{lstlisting}[style={scheme}]
(comp-hyp  3 4)
	(sqrt (+ (* 3 3) (* 4 4)))
		(+ (* 3 3) (* 4 4))
			(* 3 3)
				3 %\as% 3
				3 %\as% 3
				* %\at% 3 и 3
				9
			(* 4 4)
				4 %\as% 4
				4 %\as% 4
				* %\at% 4 и 4
				16
		+ %\at% 9 и 16
		25
	sqrt %\at% 25
	5
\end{lstlisting}
\clearpage
\section*{Задание 3}
Написать функцию, вычисляющую объем параллелепипеда по 3-м его сторонам, и
составить диаграмму ее вычисления.

\FloatBarrier
\begin{lstinputlisting}[style={lsp}]{src/3.lsp}
\end{lstinputlisting}
\FloatBarrier

\begin{lstlisting}[style={scheme}]
(volume-par 3 4 5)
	(* 3 4 5)
		3 %\as% 3
		4 %\as% 4
		5 %\as% 5
		* %\at% 3, 4, 5
		60
\end{lstlisting}

\section*{Задание 4}
Каковы результаты вычисления следующих выражений?(объяснить возможную ошибку и
варианты ее устранения)

\FloatBarrier
\begin{lstinputlisting}[style={lsp}]{src/4.lsp}
\end{lstinputlisting}
\FloatBarrier

\section*{Задание 5}
Написать функцию longer\_then от двух списков-аргументов, которая возвращает Т, если
первый аргумент имеет большую длину.

\FloatBarrier
\begin{lstinputlisting}[style={lsp}]{src/5.lsp}
\end{lstinputlisting}
\FloatBarrier

\section*{Задание 6}
Каковы результаты вычисления следующих выражений?

\FloatBarrier
\begin{lstinputlisting}[style={lsp}]{src/6.lsp}
\end{lstinputlisting}
\FloatBarrier

\section*{Задание 7}
Дана функция (defun mystery (x) (list (second x) (first x))).
Какие результаты вычисления следующих выражений? 

\FloatBarrier
\begin{lstinputlisting}[style={lsp}]{src/7.lsp}
\end{lstinputlisting}
\FloatBarrier


\section*{Задание 8}
Написать функцию, которая переводит температуру в системе Фаренгейта
температуру по Цельсию?

Как бы назывался роман Р.Брэдбери "+451 по Фаренгейту" в системе по Цельсию?

\FloatBarrier
\begin{lstinputlisting}[style={lsp}]{src/8.lsp}
\end{lstinputlisting}
\FloatBarrier

\section*{Задание 9}
Что получится при вычисления каждого из выражений?

\FloatBarrier
\begin{lstinputlisting}[style={lsp}]{src/9.lsp}
\end{lstinputlisting}
\FloatBarrier

\chapter*{Дополнительные задания}

\section*{Задание 1}
Написать функцию, вычисляющую катет по заданной гипотенузе и другому катету
прямоугольного треугольника, и составить диаграмму ее вычисления

\FloatBarrier
\begin{lstinputlisting}[style={lsp}]{src/10.lsp}
\end{lstinputlisting}
\FloatBarrier


\section*{Задание 2}
Написать функцию, вычисляющую площадь трапеции по ее основаниям и
высоте, и составить диаграмму ее вычисления.

\FloatBarrier
\begin{lstinputlisting}[style={lsp}]{src/11.lsp}
\end{lstinputlisting}
\FloatBarrier


\chapter*{Теоретические вопросы}

\section*{Базис языка \texttt{Lisp}}
Базис - минимальный набор конструкций языка и структур данных, который позволяет решить любую задачу.

Базис в Lisp образуют:
\begin{itemize}
	\item атомы;
	\item структуры;
	\item базовые функции;
	\item функционалы.
\end{itemize}

\section*{Классификация функций языка {\texttt{Lisp}}}

Функции в языке {\texttt{Lisp}}:
\begin{itemize}
	\item Базовые/чистые функции -- фиксированное кол-во аргументов, для определенного набора аргументов один фиксированный результат.
	\item Формы - функции, которые принимают произвольное количество аргументов или по разному обрабатывают аргументы.
	\item Функционалы (высшего порядка) - в качестве аргумента принимают функцию или возвращают функцию.
\end{itemize}

\section*{Способы создания функций}

С помощью макро определения \texttt{defun} или с использованием Лямбда-нотации (функция без имени).

\section*{Функции {\texttt{car, cdr}}}

Являются базовыми функциями доступа к данным. 

{\texttt{car}} принимает точечную пару или список в качестве аргумента и возвращает первый элемент или {\texttt{Nil}}.

{\texttt{cdr}} -- возвращает все элементы, кроме первого или {\texttt{Nil}}.

\section*{Функции {\texttt{list, cons}}}

Являются функциями создания списков ({\texttt{cons}} -- базовая, {\texttt{list}} -- нет). 

{\texttt{cons}} создаёт списочную ячейку и устанавливает два указателя на аргументы. 

{\texttt{list}} принимает переменное число аргументов и возвращает список, элементами которого являются аргументы, переданные в функцию.
