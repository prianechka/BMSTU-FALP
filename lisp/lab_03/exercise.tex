\chapter*{Практическое задание}

\section*{Задание 1}
Написать функцию, которая принимает целое число и возвращает первое
четное число, не меньшее аргумента.

\FloatBarrier
\begin{lstinputlisting}[style={lsp}]{src/1.lsp}
\end{lstinputlisting}
\FloatBarrier

\section*{Задание 2}
Написать функцию, которая принимает число и возвращает число
того же знака, но с модулем на 1 больше модуля аргумента.

\FloatBarrier
\begin{lstinputlisting}[style={lsp}]{src/2.lsp}
\end{lstinputlisting}
\FloatBarrier

\section*{Задание 3}
Написать функцию, которая принимает два числа и возвращает
список из этих чисел, расположенный по возрастанию.

\FloatBarrier
\begin{lstinputlisting}[style={lsp}]{src/3.lsp}
\end{lstinputlisting}
\FloatBarrier

\clearpage

\section*{Задание 4}
Написать функцию, которая принимает три числа и возвращает Т только
тогда, когда первое число расположено между вторым и третьим.

\FloatBarrier
\begin{lstinputlisting}[style={lsp}]{src/4.lsp}
\end{lstinputlisting}
\FloatBarrier

\section*{Задание 5}
Каков результат вычисления следующих выражений?

\FloatBarrier
\begin{lstinputlisting}[style={lsp}]{src/5.lsp}
\end{lstinputlisting}
\FloatBarrier

\section*{Задание 6}
Написать предикат, который принимает два числа-аргумента и возвращает
Т, если первое число не меньше второго.

\FloatBarrier
\begin{lstinputlisting}[style={lsp}]{src/6.lsp}
\end{lstinputlisting}
\FloatBarrier

\clearpage

\section*{Задание 7}
Какой из следующих двух вариантов предиката ошибочен и почему?

\FloatBarrier
\begin{lstinputlisting}[style={lsp}]{src/7.lsp}
\end{lstinputlisting}
\FloatBarrier


\section*{Задание 8}
Решить задачу 4, используя для ее решения конструкции
IF, COND, AND/OR.

\FloatBarrier
\begin{lstinputlisting}[style={lsp}]{src/8.lsp}
\end{lstinputlisting}
\FloatBarrier

\clearpage

\section*{Задание 9}
Переписать функцию how-alike, приведенную в лекции и использующую COND, используя
только конструкции IF, AND/OR.

\FloatBarrier
\begin{lstinputlisting}[style={lsp}]{src/9.lsp}
\end{lstinputlisting}
\FloatBarrier


\chapter*{Теоретические вопросы}

\section*{Базис языка \texttt{Lisp}}
Базис - минимальный набор конструкций языка и структур данных, который позволяет решить любую задачу.

Базис в Lisp образуют:
\begin{itemize}
	\item атомы;
	\item структуры;
	\item базовые функции;
	\item функционалы.
\end{itemize}

\section*{Классификация функций языка {\texttt{Lisp}}}

Функции в языке {\texttt{Lisp}}:
\begin{itemize}
	\item Базовые/чистые функции -- фиксированное кол-во аргументов, для определенного набора аргументов один фиксированный результат.
	\item Формы - функции, которые принимают произвольное количество аргументов или по разному обрабатывают аргументы.
	\item Функционалы (высшего порядка) - в качестве аргумента принимают функцию или возвращают функцию.
\end{itemize}

\section*{Способы создания функций}

С помощью макро определения \texttt{defun} или с использованием Лямбда-нотации (функция без имени).

\section*{Работа функций \texttt{and}, \texttt{or}, \texttt{if}, \texttt{cond}}

\subsection*{Функция \texttt{and}}

Синтаксис: \code{(and expression-1 expression-2 ... expression-n)}

Функция возвращает первое \texttt{expression}, результат вычисления которого = \texttt{Nil}. Если все не \texttt{Nil}, то возвращается результат вычисления последнего выражения.

\subsection*{Функция \texttt{or}}

Синтаксис: \code{(or expression-1 expression-2 ... expression-n)}

Функция возвращает первое \texttt{expression}, результат вычисления которого не \texttt{Nil}. Если все \texttt{Nil}, то возвращается \texttt{Nil}.

\subsection*{Функция \texttt{if}}

Синтаксис: \code{(if condition t-expression f-expression)}

Если вычисленный предикат не \texttt{Nil}, то выполняется \texttt{t-expression}, иначе - \texttt{f-expression}.

\subsection*{Функция \texttt{cond}}

По порядку вычисляются и проверяются на равенство с \texttt{Nil} предикаты. Для первого предиката, который не равен \texttt{Nil}, вычисляется находящееся с ним в списке выражение и возвращается его значение. Если все предикаты вернут \texttt{Nil}, то и \texttt{cond} вернет \texttt{Nil}.