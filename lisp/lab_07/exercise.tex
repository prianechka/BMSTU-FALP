\chapter*{Практическое задание}

\section*{Задание 1}
Написать хвостовую рекурсивную функцию my-reverse, которая развернет верхний
уровень своего списка-аргумента lst.

\FloatBarrier
\begin{lstinputlisting}[style={lsp}]{src/1.lsp}
\end{lstinputlisting}
\FloatBarrier

\section*{Задание 2}
Написать функцию, которая возвращает первый элемент списка-аргумента, который сам
является непустым списком.

\FloatBarrier
\begin{lstinputlisting}[style={lsp}]{src/3.lsp}
\end{lstinputlisting}
\FloatBarrier

\section*{Задание 3}
Написать функцию, которая выбирает из заданного списка только те числа, которые
больше 1 и меньше 10. (Вариант: между двумя заданными границами. )

\FloatBarrier
\begin{lstinputlisting}[style={lsp}]{src/4.lsp}
\end{lstinputlisting}
\FloatBarrier


\section*{Задание 4}
Напишите рекурсивную функцию, которая умножает на заданное число-аргумент все числа
из заданного списка-аргумента, когда

\begin{itemize}
	\item все элементы списка --- числа;
	\item элементы списка -- любые объекты.
\end{itemize}

\FloatBarrier
\begin{lstinputlisting}[style={lsp}]{src/7.lsp}
\end{lstinputlisting}
\FloatBarrier

\section*{Задание 5}
Напишите функцию, select-between, которая из списка-аргумента, содержащего только
числа, выбирает только те, которые расположены между двумя указанными границами-
аргументами и возвращает их в виде списка (упорядоченного по возрастанию списка чисел
(+ 2 балла)).

\FloatBarrier
\begin{lstinputlisting}[style={lsp}]{src/8.lsp}
\end{lstinputlisting}
\FloatBarrier

\section*{Задание 6}
Написать рекурсивную версию (с именем rec-add) вычисления суммы чисел заданного списка:

\begin{itemize}
	\item одноуровнего смешанного;
	\item структурированного.
\end{itemize}

\FloatBarrier
\begin{lstinputlisting}[style={lsp}]{src/81.lsp}
\end{lstinputlisting}
\FloatBarrier

\section*{Задание 7}
Написать рекурсивную версию с именем recnth функции nth.

\FloatBarrier
\begin{lstinputlisting}[style={lsp}]{src/9.lsp}
\end{lstinputlisting}
\FloatBarrier


\section*{Задание 8}
Написать рекурсивную функцию allodd, которая возвращает t когда все элементы списка
нечетные.

\FloatBarrier
\begin{lstinputlisting}[style={lsp}]{src/10.lsp}
\end{lstinputlisting}
\FloatBarrier


\section*{Задание 9}
Написать рекурсивную функцию, которая возвращает первое нечетное число из списка
(структурированного), возможно создавая некоторые вспомогательные функции.

\FloatBarrier
\begin{lstinputlisting}[style={lsp}]{src/11.lsp}
\end{lstinputlisting}
\FloatBarrier

\section*{Задание 10}
Используя cons-дополняемую рекурсию с одним тестом завершения,
написать функцию которая получает как аргумент список чисел, а возвращает список
квадратов этих чисел в том же порядке.

\FloatBarrier
\begin{lstinputlisting}[style={lsp}]{src/12.lsp}
\end{lstinputlisting}
\FloatBarrier