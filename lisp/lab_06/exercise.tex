\chapter*{Практическое задание}

\section*{Задание 1}
Напишите функцию, которая уменьшает на 10 все числа из списка-аргумента этой
функции.

\FloatBarrier
\begin{lstinputlisting}[style={lsp}]{src/1.lsp}
\end{lstinputlisting}
\FloatBarrier

\section*{Задание 2}
Напишите функцию, которая умножает на заданное число-аргумент все числа
из заданного списка-аргумента, когда
\begin{itemize}
	\item все элементы списка --- числа;
	\item элементы списка -- любые объекты.
\end{itemize}

\FloatBarrier
\begin{lstinputlisting}[style={lsp}]{src/2.lsp}
\end{lstinputlisting}
\FloatBarrier

\section*{Задание 3}
Написать функцию, которая по своему списку-аргументу lst определяет
является ли он палиндромом (то есть равны ли lst и (reverse lst)).

\FloatBarrier
\begin{lstinputlisting}[style={lsp}]{src/3.lsp}
\end{lstinputlisting}
\FloatBarrier


\section*{Задание 4}
Написать предикат set-equal, который возвращает t, если два его множества-
аргумента содержат одни и те же элементы, порядок которых не имеет значения.

\FloatBarrier
\begin{lstinputlisting}[style={lsp}]{src/4.lsp}
\end{lstinputlisting}
\FloatBarrier

\section*{Задание 5}
Написать функцию которая получает как аргумент список чисел, а возвращает список
квадратов этих чисел в том же порядке.

\FloatBarrier
\begin{lstinputlisting}[style={lsp}]{src/5.lsp}
\end{lstinputlisting}
\FloatBarrier

\section*{Задание 6}
Напишите функцию, select-between, которая из списка-аргумента, содержащего только
числа, выбирает только те, которые расположены между двумя указанными границами-
аргументами и возвращает их в виде списка (упорядоченного по возрастанию списка чисел
(+ 2 балла)).

\FloatBarrier
\begin{lstinputlisting}[style={lsp}]{src/6.lsp}
\end{lstinputlisting}
\FloatBarrier

\clearpage

\section*{Задание 7}
Написать функцию, вычисляющую декартово произведение двух своих списков-
аргументов. ( Напомним, что А х В это множество всевозможных пар (a b), где а
принадлежит А, принадлежит В.)

\FloatBarrier
\begin{lstinputlisting}[style={lsp}]{src/7.lsp}
\end{lstinputlisting}
\FloatBarrier


\section*{Задание 8}
Почему так реализовано reduce, в чем причина?

\FloatBarrier
\begin{lstinputlisting}[style={lsp}]{src/8.lsp}
\end{lstinputlisting}
\FloatBarrier

В первом случае сумма всех элементов равна 0, а во втором случае -- пустой массив, 
следовательно, в нём нет элементов, значит, сумма всех элементов равна 0.

\section*{Задание 9}
Пусть list-of-list список, состоящий из списков. Написать функцию, которая вычисляет
сумму длин всех элементов list-of-list, т.е. например для аргумента ((1 2) (3 4)) -> 4.

\FloatBarrier
\begin{lstinputlisting}[style={lsp}]{src/9.lsp}
\end{lstinputlisting}
\FloatBarrier